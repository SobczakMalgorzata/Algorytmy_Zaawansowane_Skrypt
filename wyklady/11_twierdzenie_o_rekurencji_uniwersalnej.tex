% z przykładem stosowania rekurencji do mnożnia macieży 
% na przykładzie algorytmu Strassena 
Niech $a \geq 1, b > 1$, będą stałymi, $f(n) = \Theta(n^k)$\\
Niech: $T(n) - aT(n/b)+f(n)$ $(n/b$ oznacza $\lfloor n/b \rfloor $ lub $\lceil n/b \rceil)$\\
\\
wtedy 
\[
  T(n)=\begin{cases}
               \Theta(n^{log_ba}),\ dla\ a>b^k \Leftrightarrow log_ba>k\\
               \Theta(n^{log_ba}logn),\ dla\ a=b^k \Leftrightarrow log_ba=k\\
               \Theta(n^k),\ dla\ a<b^k \Leftrightarrow log_ba<k
            \end{cases}
\]\\
$a^{log_Bn}=b^{log_ba^{log_bn}}=b^{log_bnlog_ba}=b^{log_balog_bn}=b^{log_ba^{log_bn}}=n^{log_ba}$\\
\\
\textbf{Lemat: }Niech $a\geq 1,\ b>1$ będą stałymi, niech f(n) będzie funkcją nieujemną zdefiniowaną \\dla potęg b.\\
Jeśli 
\[
  T(n)=\begin{cases}
               \Theta(1),\ dla\ n=1\\
               aT(n/b) + f(n),\ dla\ n=b^i,\ gdzie\ i \in {\rm I\!N}_+
            \end{cases}
\]\\
to\\
$T(n) = \Theta(n^{log_ba}) + \sum_{j=0}^{log_bn-1} a^j\ f(n/b^j)$\\
\\
\textbf{Dowód: }
$T(n)=f(n) +a T(n/b)= f(n) + a(f(n/b)+a\ T(n/b^2)) =f(n)+a\ f(n/b) + a^2\ T(n/b^2) = $\\
$=f(n) + a\ f(n/b) + a^2\ f(n/b^2) + a^3\ T(n/b^3) = ... =$\\
$= f(n) + a\ f(n/b) + a^2\ f(n/b^2) +...+a^{log_bn-1}T(\frac{n}{b^{log_bn-1}})+ a^{log_bn}T(1)$\\
\\
ponieważ $a^{log_bn}=n^{log_ba}$ stąd $a^{log_bn}T(1) = \Theta(n^{log_ba})$\\
oraz $T(n) = \Theta(n^{log_ba})+\sum_{j=0}^{log_bn-1} a^j\ f(n/b^j)$ \\
\\
\textbf{Dowód: }twierdzenia dla przypadku gdy n jest potęgą b (w pozostałych przypadkach dowód jest bardziej techniczny)\\
\\
$\Theta(\sum_{j=0}^{log_bn-1} a^j\ f(n/b^j))=\Theta(\sum_{j=0}^{log_bn-1} a^j\ (n/b^j)^k)$\\
\begin{enumerate}
\item Jeśli $a>b^k$, a $log_ba>k$\\
	\\
	$\sum_{j=0}^{log_bn-1} a^j\ (n/b^j)^k = \sum_{j=0}^{log_bn-1} a^j\ \frac{n^k}{b^{jk}}=n^k\ \sum_{j=0}^{log_bn-1} (\frac{a}{b^k})^j = n^k\ \frac{1-(\frac{a}{b^k})^{log_bn}}{1-\frac{a}{b^k}}= n^k\ \frac{\frac{a^{log_bn}}{b^{klog_bn}}-1}{\frac{a}{b^k}-1}=$\\
	$n^k\ \frac{\frac{a^{log_bn}-b^{klog_bn}}{b^{klog_bn}}}{\frac{a-b^k}{b^k}}= n^k\ \frac{a^{log_ba}-b^{log_Bn^k}}{a-b^k} \ast \frac{b^k}{b^{log_bn^k}}= n^k\ \frac{n^{log_ba}-n^k}{a\ast b^k}\ast \frac{b^k}{n^k} = \frac{b^k}{a-b^k}(n^{log_ba}-n^k) = \Theta(n^{log_ba})$\\
	\\
	z lematu $T(n) = \Theta(n^{log_ba})$
\item Jeśli $a=b^k$, a $log_ba=k$\\
	\\
	$\sum_{j=0}^{log_bn-1} a^j\ (n/b^j)^k = n^k\sum_{j=0}^{log_bn-1} )\frac{a}{b^k})^j = n^k\ \sum_{j=0}^{log_bn-1} 1 = n^k \ast lob_bn$\\
	\\
	Zatem $T(n) = \Theta(n^{log_ba}) + \Theta(n^klog)bn) = \Theta(n^{log_bn} + n^{log_ba}log_bn) = \Theta(n^{log_ba}log_bn)$
\item Jeśli $a<b^k$, a $log_ba<k$\\
	\\
	$\sum_{j=0}^{log_bn-1} a^j\ (n/b^j)^k = $ jak w pierwszym $ = n^k\ \frac{n^k-n^{log_ba}}{b^k-a} \ast \frac{b^k}{n^k} = \frac{b^k}{b^k-a}\ast (n^k \ast n^{log_ba})=\Theta(n^k)$\\
	\\
	Z lematu $T(n) = \Theta(n^{log_ba}) + \Theta(n^k) = O(n^k)$
\end{enumerate}

\textbf{Wracamy do problemu mnożenia liczb n-cyfrowych}\\
\\
\tab $T(n) = 4T(\frac{n}{2}) + O(n)$\\
$a=4$\tab $b=2$\tab $k=1$\tab $4>2^1$\\
Więc złożoność:\\
\tab$T(n) = \Theta(n^{log_24})=\Theta(n^2)$\\
Czyli tak jak mnożenie pisemne\\
\\
Dla sprytnego algorytmu: \\
\tab $T(n) = 3T(\frac{n}{2})+O(n)$\\
$a=3$\tab $b=2$\tab $k=1$\tab $3>2^1$\\
\tab $T(n)= \Theta(n^{log_23})=O(n^{1,59})$\\
\\
\textbf{Przykład: Mnożenie Macierzy}\\
Zwykły algorytm mnożenia macierzy nxn działa w czasie $O(n^3)$\\
\\
$A\ast B=C$ A, B $=$ macierze nxn\\
Dzielimy A i B na kwadraty\\
\\
\[
A\ast B = 
\begin{bmatrix}
    A_{11}       & A_{12} \\
    A_{21}       & A_{22} 
\end{bmatrix}\ 
\begin{bmatrix}
    B_{11}       & B_{12} \\
    B_{21}       & B_{22} 
\end{bmatrix}
=
\begin{bmatrix}
    C_{11}       & C_{12} \\
    C_{21}       & C_{22} 
\end{bmatrix} = C
\]
\\
gdzie $C_{11}= A_{11}\ast B_{11}+A_{12}\ast B_{21}$\\
\tab $C_{12}= A_{11}\ast B_{12}+A_{21}\ast B_{22}$\\
\tab $C_{21}= A_{21}\ast B_{11}+A_{22}\ast B_{21}$\\
\tab $C_{22}= A_{21}\ast B_{12}+A_{22}\ast B_{22}$\\
\\
Zamieniamy jedno mnożenie macierzy nxn na 8 mnożeń macierzy $\frac{n}{2}$x$\frac{n}{2}$ kosztujące $O(n^2)$ czasu.\\
\\
\tab $T(n)=8T( \frac{n}{2} )+ O(n^2)$\\
\\
$a=8$\tab $b=2$\tab $k=2$\tab $8>2^2$\\
\\
\tab $T(n)=\Theta(n^{log_28} = \Theta(n^3)$\\

\textbf{Algorytm Strassena}\\
\\
Strassen zauważył,że można policzyć:\\
$M_1=(A_{12}-A_{22})\ast (B_{21}+B_{22})$\\
$M_2=(A_{11}-A_{22})\ast (B_{11}+B_{22})$\\
$M_3=(A_{11}-A_{21})\ast (B_{11}+B_{12})$\\
$M_4=(A_{11}-A_{12})\ast B_{22}$\\
$M_5=A_{11}\ast (B_{12}-B_{22})$\\
$M_6=A_{22}\ast (B_{21}-B_{11})$\\
$M_7=(A_{21}+A_{22})\ast B_{11}$\\
\\
i wtedy: \\
\tab $C_{11}=M_1+M_2-M_4+M_6$\\
\tab $C_{12}=M_4+M_5$\\
\tab $C_{21}=M_6+M_7$\\
\tab $C_{22}=M_2-M_3+M_5-M_7$\\
\\
Stąd rekurencja:\\
\tab $T(n)=7T(\frac{n}{2})+O(n^2)$\\
$a=7$\tab $b=2$\tab $k=2$\tab $7>2^2$\\
\tab $T(n)= \Theta(n^{log_27})=O(n^{2,81})$\\