\textbf{Algorytm aproksymacyjny (przybliżony)} - daje "prawie optymalne" rozwiązanie\\
\\
Stosuje się algorytmy  aproksymacyjne tam, gdzie jest mała szansa na wielomianowy algorytm dokładny\\
\\
$\Pi$ - problem aproksymacyjny polegający na maksymalizacji pewnej funkcji celu.\\\\
Niech "$c^{\ast}$" - optymalne rozwiązanie $\Pi$, $c^{\ast} > 0$\\
\tab $c$ - wartość znaleziona przez algorytm aproksymacyjny A dla $\Pi$, $c>0$\\\\
Mówimy, że algorytm aproksymacyjny A ma ograniczenie względne (współczynnik aproksymacji) $\rho (n)$ jeśli dla dowolnych danych o rozmiarze $n$ $max(\frac{c}{c^{\ast}},\frac{c^{\ast}}{c}) \leq \rho (n)$. \\\\
A nazywamy algorytmem $\rho(n)$ - aproksymacyjnym \\
$\begin{drcases}
dla\ problemu\ maksymalizacji\ 0<c\leq c^{\ast}\\
dla\ problemu\ minimalizacji\ 0<c^{\ast}\leq c
\end{drcases} \Rightarrow max(\frac{c}{c^{\ast}},\frac{c^{\ast}}{c})\geq 1 \Rightarrow \rho(n) \geq 1$\\\\
Błąd względny rozwiązania $c$ to $\frac{\vert c-c^{\ast}}{c^{\ast}}$\\\\
Mówimy, że algorytm aproksymacyjny  A posiada ograniczenia błędu  względnego $\varepsilon (n)$ jeśli dla danych o rozmiarze $n$ $\frac{\vert c-c^{\ast}\vert}{c^{\ast}} \leq \varepsilon (n)$ \\
dla problemu maksymalizacji $\frac{\vert c^{\ast} - c\vert}{c^{\ast}} = \frac{c^{\ast}}{c} - 1 \leq \rho(n) - 1 \tab 0<c\leq c^{\ast}$\\
dla problemu maksymalizacji $\frac{\vert c - c^{\ast}\vert}{c^{\ast}} = \frac{c - c^{\ast}}{c} = \frac{c}{c^{\ast}} - 1 \leq \rho(1) \tab 0<c^{\ast}\leq c$\\\\
\begin{center}
$\rho = sup_{n \in {\rm I\!N}}\lbrace\rho (n) \rbrace$\tab $\varepsilon = sup_{n \in {\rm I\!N}}\lbrace\varepsilon (n) \rbrace$
\end{center}
\textbf{Mówimy wtedy, że współczynnik aproksymacji wynosi $\rho$ i że A jest algorytmem $\rho$-aproksymacyjnym.}