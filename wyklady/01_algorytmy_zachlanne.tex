\subsubsection{Notacja}
\(f,g,{\rm I\!N} \rightarrow {\rm I\!R}_+\)
\begin{itemize}
	\item \( g(n) = O(f(n))\) if \(\exists{c}>0\) \(\exists{n_0}\in {\rm I\!N} \) \(\forall{n} \leq n_0\) \(g(n) \geq c f(n)\)
    \item \( g(n) = \Omega(f(n))\) if \(\exists{c}>0\) \(\exists{n_0}\in {\rm I\!N} \) \(\forall{n} \leq n_0\) \(cf(n) \geq g(n)\)
    \item \( g(n) = \Theta(f(n))\) if \(\exists{c_1,c_2}>0\) \(\exists{n_0}\in {\rm I\!N} \) \(\forall{n} \leq n_0\) \(c_1 f(n)\geq g(n) \geq c_2 f(n)\)
\end{itemize}
\subsubsection{Algorytmy Zachłanne}
Idea: w każdym kroku algorytmu dokonuje lokalnie najlepszego wyboru, czyli wyboru, który w danym momencie wydaje się najkorzystniejszy.